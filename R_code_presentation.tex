\documentclass{beamer}
\usepackage{graphicx} % Required for inserting images
% usetheme function to change layout
\usetheme{Frankfurt}
%
%
%change colour of theme
\usecolortheme{seahorse}
%
\title{Spacial analysis of Montello area}
\author{Marion Anna}
\date{July 2025}

\begin{document}

\maketitle


\section{Introduction}


\begin{frame}{Aim of the project}
The aim of the project is to analyze the Montello area, in the Treviso province  of the Veneto region. Specifically, the area of "Grave di Ciano" is regard \begin{itemize}
        \item Grave di Ciano \\
        10.82428, 11.80481, 43.71259, 44.15045 (3831 km2)\\
        - Comparison 2016-2024
        \bigskip
        \item Bilancino Lake \\
        11.20228, 11.42716, 43.92792, 44.04765 (240 km2)\\
        - Comparison December-July 2023
        \bigskip
        \item Brento Cave of Serena stone \\
        11.32187, 11.52237, 44.08956, 44.1792 (160 km2) \\
        - Biodiversity analysis
        \bigskip
    \end{itemize}
\end{frame}

\begin{frame}{Data Collection}
Sentinel 2 - Copernicus Open Hub
\begin{itemize}
    \item Select area of interest
    \item Select dates of interest
    \item Analytical image
    \item Image format: JPG (no georeference)
    \item Coordinate system: WGS 84 (EPSG:4326) 
    \item Download of all raw layers
\end{itemize}
\end{frame}

\begin{frame}{Libraries}
To complete the project the following libraries were used:
    \begin{itemize}
        \item terra
        \item imageRy
        \item ggplot2
        \item viridis
    \end{itemize}
\end{frame}


\section{Grave di Ciano}


\begin{frame}{Grave di Ciano}
    \begin{columns}
        \column{0.5\textwidth}
        \centering
        \includegraphics[width=1\textwidth]{Map_-_IT_-_Firenze_-_Comunità_Montana_Mugello.svg.png}
         
        \column{0.5\textwidth}
        \centering
        \includegraphics[width=1\textwidth]{mugello.png}
    \end{columns} 
\end{frame}

\begin{frame}{Import of the 4 bands B2, B3, B4, B8}
    \centering
Function rast() from terra
    \begin{columns}
        \column{0.5\textwidth}
        \centering
        \includegraphics[width=1\textwidth]{Bends 2016.png}
        \caption{2016}
         
        \column{0.5\textwidth}
        \centering
        \includegraphics[width=1\textwidth]{Bands 2024.png}
        \caption{2024}
    \end{columns} 
\end{frame}

\begin{frame}{Stack of the four bands}
     \centering
    c(B2, B3, B4, B8)
    \includegraphics[width=1\textwidth]{Stackvir.png}
\end{frame}

\begin{frame}{True color}
     \centering
     im.plotRGB(stack, 3, 2, 1)
    \includegraphics[width=1\textwidth]{Mreality.png}
\end{frame}

\begin{frame}{False color}
     \centering
     im.plotRGB(stack, 4, 3, 2)
    \includegraphics[width=1\textwidth]{Mfalse.png}
\end{frame}

\begin{frame}{SWIR (Short Wave Infrared)}
     \centering
     Different stack :  B12, B8A, B04 \\
     im.plotRGB(stack, 3, 2, 1)
    \includegraphics[width=1\textwidth]{Mswir.png} 
\end{frame}


\section{2022 Drought}


\begin{frame}{2022 Drought}
     \centering
     Area (11.20228, 11.42716, 43.92792, 44.04765)
    \includegraphics[width=1\textwidth]{Bilancino.png}
\end{frame}

\begin{frame}{NDWI (Normalized Difference Water Index) }
     \centering
     \begin{equation}
        NDWI = \frac{B03 - B08}{B03 +B08}
     \end{equation}
    \includegraphics[width=1\textwidth]{Rplot.png}
\end{frame}

\begin{frame}{Difference }
     \centering
    \includegraphics[width=1\textwidth]{NDWIdiff.png}
\end{frame}

\begin{frame}{Classification}
    \centering
    \begin{columns}
        \column{0.5\textwidth}
        \centering
        July
        \smallskip
        \includegraphics[width=1\textwidth]{lcalss.png}
        \includegraphics[width=1\textwidth]{Graphj.png}
         
        \column{0.5\textwidth}
        \centering
        December
        \smallskip
        \includegraphics[width=1\textwidth]{dclass.png}
        \includegraphics[width=1\textwidth]{Graphd.png}
    \end{columns} 
\end{frame}


\section{Variability analysis}


\begin{frame}{Variability analysis}
     \centering
     Area (11.32187, 11.52237, 44.08956, 44.1792)
    \includegraphics[width=1\textwidth]{cave.png}
\end{frame}

 \begin{frame}{NDVI (Normalized Difference Vegetation Index) }
     \centering
     \begin{equation}
        NDVI = \frac{B08 - B04}{B08 + B04}
     \end{equation}
     \smallskip
    \includegraphics[width=0.7\textwidth]{ndvinew.png}
\end{frame}

\begin{frame}{Principal component analysis PCA}
    im.pca() on stack of the four bands B02, B03, B04, B08
    \centering
    \includegraphics[width=0.9\textwidth]{pca.png} 
\end{frame}

\begin{frame}{Variability}
    focal(pc1, matrix(1/9, 3, 3), sd)
    \centering
    \bigskip
    \includegraphics[width=\textwidth]{varmagma.png}   
\end{frame}



\end{document}
