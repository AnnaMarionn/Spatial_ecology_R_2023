\documentclass{beamer}
\usepackage{graphicx} % Required for inserting images
% usetheme function to change layout

\usetheme{Frankfurt}
%
%
%change colour of theme
\usecolortheme{seahorse}
%
\title{Spacial analysis of Montello area}
\author{Marion Anna}
\date{July 2025}

\begin{document}

\maketitle


\section{Introduction}

\begin{frame}{Area: Montello and Grave di Ciano}
The Montello hill is a key natural landmark in Veneto, characterized by diverse vegetation and land use patterns. In particular, the Grave di Ciano is a \textbf{protected riverine area}, part of the Natura 2000 network, hosting rare habitats and high biodiversity. With an extension of about 940 hectares, it includes dry grasslands, riparian woodlands, and wet habitats. This area is currently \textbf{threatened} by a proposed flood retention basin project.

\end{frame}

\begin{frame}
\includegraphics[width=1\textwidth]{Real_colors_2024_grave.png}
\end{frame}

\begin{frame}{Aim of the project}
This study aims to:
\begin{itemize}
        \item Land cover change \\
        Analyze land cover changes over recent years in the Montello area.\\
        - Comparison 2016-2023
        \bigskip
        \item 2022 drought \\
        Assess the impact of the 2022 drought on vegetation health in Grave di Ciano.\\
        - Comparison summer 2019-2022
        \bigskip
        \item Variability analysis \\
        Investigate spatial variability and heterogeneity of vegetation in the threatened area \\
        - Summer 2025
        \bigskip
    \end{itemize}
\end{frame}

\begin{frame}{Data Collection}
Sentinel 2 - Copernicus Open Hub
\begin{itemize}
    \item Select area of interest
    \item Select dates of interest
    \item Image format: jp2
    \item Coordinate system: TM Zone 32N (EPSG:32632)
    \item Download of all raw bands
\end{itemize}
\end{frame}

\begin{frame}{Libraries}
To complete the project the following libraries were used:
    \begin{itemize}
        \item terra
        \item imageRy
        \item ggplot2
        \item viridis
    \end{itemize}
\end{frame}


\section{Land cover change}


\begin{frame}{False color comparison}
        \centering
        Plotting false color view to enhance vegetation:
        \includegraphics[width=1\textwidth]{FC_comparison.png}
\end{frame}


\begin{frame}{Classification using NDVI}
     \centering
        \begin{equation}
        NDVI=\frac{NIR-RED}{NIR+RED}
        \end{equation} 
    \includegraphics[width=1\textwidth]{NDVIClassification2016-2023.png}
\end{frame}

\begin{frame}{Land cover change}
     \centering
    \includegraphics[width=1\textwidth]{Land_cover_change_stackedbarplot.png}
\end{frame}

\section{2022 Drought}


\begin{frame}{Grave di Ciano- Bands}
     \centering
    \includegraphics[width=1\textwidth]{2024_bands.png}
\end{frame}

\begin{frame}{Real color comparison}
     \centering
    \includegraphics[width=1\textwidth]{June-August_2019-2022.png}
\end{frame}

\begin{frame}{Composite SWIR - 2022}
     \centering
    \includegraphics[width=1\textwidth]{swir_jun_aug.png}
\end{frame}


\begin{frame}{NDWI (Normalized Difference Water Index) }
     \centering
     Gao's NDWI (1996). Comparison August 2019-2022
     \begin{equation}
        NDWI = \frac{NIR - SWIR2}{NIR +SWIR2}
     \end{equation}
    \includegraphics[width=1\textwidth]{NDWIDiff_19-22.png}
\end{frame}

\section{Variability analysis}


\begin{frame}{Bands comparison}
     \centering
    \includegraphics[width=1\textwidth]{2025_pairs.png}
\end{frame}


\begin{frame}{Standard Deviation and PCA}
    \centering
    Standard deviation calculated on NIR band and on PC1 through moving window (focal sd).
    \includegraphics[width=0.9\textwidth]{SD_analysis.png} 
\end{frame}

\begin{frame}{Thank you for your attention!
 My GitHub profile:\url{https://github.com/AnnaMarionn}
     \centering}
    
\end{frame}



\end{document}
